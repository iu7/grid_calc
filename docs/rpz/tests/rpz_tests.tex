% !TeX document-id = {fb8a2ef5-cdaf-49da-b79d-0a8152e677cd}
% !TeX TS-program = XeLaTeX
\documentclass[a4paper,12pt]{report}

% polyglossia should go first!
\usepackage{polyglossia} % multi-language support
\setmainlanguage{russian}
\setotherlanguage{english}

\usepackage{amsmath} % math symbols, new environments and stuff
\usepackage{unicode-math} % for changing math font and unicode symbols
\usepackage[style=english]{csquotes} % fancy quoting
\usepackage{microtype} % for better font rendering
\usepackage{hyperref} % for refs and URLs
\usepackage{graphicx} % for images (and title page)
\usepackage{geometry} % for margins in title page
\usepackage{tabu} % for tabulars (and title page)
\usepackage[section]{placeins} % for float barriers
\usepackage{titlesec} % for section break hooks
\usepackage{listings} % for listings 
\usepackage{upquote} % for good-looking quotes in source code (used for custom languages)
\usepackage{xcolor} % colors!
\usepackage{enumitem} % for unboxed description labels (long ones)
\usepackage{caption}
\usepackage{multirow}
\usepackage{varwidth}
\usepackage{longtable}

\defaultfontfeatures{Mapping=tex-text} % for converting "--" and "---"
\setmainfont{CMU Serif}
\setsansfont{CMU Sans Serif}
\setmonofont{CMU Typewriter Text}
\setmathfont{XITS Math}
\MakeOuterQuote{"} % enable auto-quotation

% new page and barrier after section, also phantom section after clearpage for
% hyperref to get right page.
% clearpage also outputs all active floats:
\newcommand{\sectionbreak}{\phantomsection}
\newcommand{\subsectionbreak}{\FloatBarrier}
\renewcommand{\thesection}{\arabic{section}} % no chapters
\numberwithin{equation}{section}
%\usetikzlibrary{shapes,arrows,trees}

\newcommand{\itemtt}[1][]{\item[\texttt{#1}:]} % tt-ed items (for protocol descriptions)

\definecolor{bluekeywords}{rgb}{0.13,0.13,1}
\definecolor{greencomments}{rgb}{0,0.5,0}
\definecolor{turqusnumbers}{rgb}{0.17,0.57,0.69}
\definecolor{redstrings}{rgb}{0.5,0,0}
\setmonofont{Consolas} %to be used with XeLaTeX or LuaLaTeX
\definecolor{bluekeywords}{rgb}{0,0,1}
\definecolor{greencomments}{rgb}{0,0.5,0}
\definecolor{redstrings}{rgb}{0.64,0.08,0.08}
\definecolor{xmlcomments}{rgb}{0.5,0.5,0.5}
\definecolor{types}{rgb}{0.17,0.57,0.68}

\lstloadlanguages{bash, python, Java}

\newcommand{\tabitem}{~~\llap{\textbullet}~~}
\newcolumntype{L}[1]{>{\raggedright\let\newline\\\arraybackslash\hspace{0pt}}m{#1}}
\newcolumntype{C}[1]{>{\centering\let\newline\\\arraybackslash\hspace{0pt}}m{#1}}
\newcolumntype{R}[1]{>{\raggedleft\let\newline\\\arraybackslash\hspace{0pt}}m{#1}}

\lstset{
  frame=none,
  xleftmargin=2pt,
  stepnumber=1,
  numbers=left,
  numbersep=5pt,
  numberstyle=\ttfamily\tiny\color[gray]{0.3},
  belowcaptionskip=\bigskipamount,
  captionpos=b,
  escapeinside={*'}{'*},
  language=python,
  tabsize=2,
  emphstyle={\bf},
  commentstyle=\it,
  stringstyle=\mdseries\rmfamily,
  showspaces=false,
  keywordstyle=\bfseries\rmfamily,
  columns=flexible,
  basicstyle=\small\sffamily,
  showstringspaces=false,
  morecomment=[l]\%,
  breaklines=true,
  showlines=true
}
\renewcommand\lstlistingname{Листинг}

\date{\today}

\makeatletter
\let\thetitle\@title
\let\theauthor\@author
\let\thedate\@date
\makeatother

\makeatletter
\AtBeginDocument{%
    \expandafter\renewcommand\expandafter\subsection\expandafter{%
        \expandafter\@fb@secFB\subsection
    }%
}
\makeatother

\begin{document}

\section{Тестирование}
\subsection{Модульное тестирование}
Статический класс DataMethods:


Представляет собой интерфейс бекенда данных.


Производится тестирование класса DataMethods с использованием методики тестирования -- разбиение на уровне класса на категории по функциональности. Категория объединяет в себе методы класса, выполняющие близкую по смыслу функциональность.


Методы класса можно разбить на 3 категории по функциональности:
\begin{itemize}
  \item методы получения данных;
  \item методы изменения данных;
  \item методы удаления данных. 
\end{itemize}

\begin{table}[h!]
\caption{Методы класса DataMethodsFilter}
\begin{tabu} to \textwidth {|c|X|}
\hline
Название метода & Примечания \\
\hline
\multirow{3}{*}{get\_item} & param: table [ str ] \\
                           & param: value\_json[ dict ] \\
                           & Возвращает все элементы таблицы с именем table, удовлетворяющие фильтру value\_json \\
\hline
\multirow{3}{*}{put\_item} & param: table [ str ] \\
                           & param: value\_json [ dict ] \\
                           & Изменяет все элементы таблицы с именем table, удовлетворяющие фильтру value\_json \\
\hline
\multirow{3}{*}{delete\_item} & param: table [ str ] \\
                           & param: value\_json [ dict ] \\
                           & Удаляет все элементы таблицы с именем table, удовлетворяющие фильтру value\_json \\
\hline
\end{tabu}
\end{table}

%%%
\clearpage
%%%

\begin{table}[h]
\caption{Категория 1 -- Тестирование метода получения данных}
\begin{tabu} to \textwidth {|c|X|}
\hline
Название теста & TestGetItemEmpty \\ \hline
Тестируемый метод & get\_item \\ \hline
Описание теста & Проверка получения данных из БД, пустой таблицы trait \\ \hline
Ожидаемый результат & Cловарь с пустым списком в ключе ``result'' \\ \hline
Степень важности & Фатальная \\ \hline
Результат теста & Тест пройден \\ \hline
\end{tabu}
\end{table}

%%%

\begin{table}[h]
\caption{Категория 1 -- Тестирование метода получения данных}
\begin{tabu} to \textwidth {|c|X|}
\hline
Название теста & TestGetItemGeneral \\ \hline
Тестируемый метод & get\_item \\ \hline
\multirow{4}{*}{Описание теста} & Проверка изменения данных в БД, в таблице trait, предварительно наполненной записями \\
                                & name = ``test\_name'', version = ``1.0'' \\
                                & name = ``test\_name'', version = ``2.0'', \\
                                & по фильтру \{``name'':``test\_name''\} \\
\hline
\multirow{3}{*}{Ожидаемый результат} & Словарь \{``result'':[\{``name'':``test\_name'', \\
                                & ``version'':``1.0''\}, \{``name'':``test\_name'',\} \\
                                & ``version'':``2.0''\}]\} \\
\hline
Степень важности & Фатальная \\ \hline
Результат теста & Тест пройден \\ \hline
\end{tabu}
\end{table}

%%%
\clearpage
%%%

\begin{table}[h]
\caption{Категория 2 -- Тестирование метода изменения данных}
\begin{tabu} to \textwidth {|c|X|}
\hline
Название теста & TestPutItem \\ \hline
Тестируемый метод & put\_item \\ \hline
\multirow{4}{*}{Описание теста} & Проверка изменения данных в БД, в таблице trait, предварительно наполненной записями \\
                                & name = ``test\_name'', version = ``1.0'' \\
                                & name = ``test\_name'', version = ``2.0'', \\
                                & по фильтру \{``name'':``test\_name'',``changes'': \{``version'':``3.0''\}\} \\
\hline
Ожидаемый результат & Словарь \{``count'': 2\}, отражающий наличие двух произведенных изменений в БД \\ \hline
Степень важности & Фатальная \\ \hline
Результат теста & Тест пройден \\ \hline
\end{tabu}
\end{table}

%%%

\begin{table}[h]
\caption{Категория 3 -- Тестирование метода удаления данных}
\begin{tabu} to \textwidth {|c|X|}
\hline
Название теста & TestDeleteItem \\ \hline
Тестируемый метод & delete\_item \\ \hline
\multirow{4}{*}{Описание теста} & Проверка удаления данных в БД, в таблице trait, предварительно наполненной записями \\
                                & name = ``test\_name'', version = ``1.0'' \\
                                & name = ``test\_name'', version = ``2.0'', \\
                                & по фильтру \{``name'':``test\_name''\} \\
\hline
Ожидаемый результат & Словарь \{``count'': 2\}, отражающий наличие двух произведенных удалений в БД \\ \hline
Степень важности & Фатальная \\ \hline
Результат теста & Тест пройден \\ \hline
\end{tabu}
\end{table}

\subsubsection{Вывод по результатам тестирования}
Все тесты пройдены успешно, класс готов к использованию.

\clearpage
\subsection{Системное тестирование}
Производится системное тестирования файлового сервера (стратегия чёрного ящика).
Тест покрывает все прецеденты взаимодействия с файловым сервером.

При формировании тестовых наборов использовалась методика эквивалентного разбиения для входных данных.
%При проведении тестирования сервер был запущен по адресу \url{localhost:50002}

Классы эквивалентности для входных данных:

\noindent
\begin{tabu}{|L{3cm}|L{5cm}|L{5cm}|}\hline
  Параметр & Допустимые классы эквивалентности & Недопустимые классы эквивалентности \\\hline
  Имена файлов & Строки, не содержащие символы \&,\$,/,:,*,? и другие спецсимволы & Строки, содержащие запрещённые символы \\\hline
  Адрес сервера & Строки вида 'http://address:port' & Строки, не подходящее под описание \\\hline
  Удалённый ресурс & Такие же строки, как и допустимые для имён файлов & Такие же строки, как и недопустимые для имён файлов \\\hline
  Запрос-файл & Запрос с содержимым файла в теле и с хедером 'Content-type':'multipart/form-data' & Запрос без необходимого заголовка либо испорченным телом \\\hline
\end{tabu}

\clearpage
Тесты:

\noindent
\begin{tabu}{|L{1cm}|L{3cm}|L{2.5cm}|L{2cm}|X[l]|}\hline
 	№ & Описание теста & Входные данные & Ожидаемый результат & Полученный результат \\ \hline
 	1 & Проверка возможности сохранения файла & Файл-запрос & Сообщение об успешном сохранении файла & Сообщение протокола HTTP 200 OK, пустой JSON-объект \\ \hline
 	2 & Проверка доступа к существующему файлу & Запрос по адресу файла & Содержимое файла & Сообщение протокола HTTP 200 OK, содержимое искомого файла \\ \hline
 	3 & Проверка удаления существующего файла & Запрос на удаление файла & Сообщение об успешном удалении файла & Сообщение протокола HTTP 200 OK, пустой JSON-объект \\ \hline
 	4 & Проверка удаления несуществующего файла & Запрос на удаление файла по несуществующему пути & Сообщение об ошибке & Сообщение протокола HTTP 404 NOT FOUND, JSON-объект с сообщением об ошибке \\ \hline
 	5 & Проверка доступа к несуществующему файлу & Запрос по адресу несуществующего файла & Сообщение об ошибке & Сообщение протокола HTTP 404 NOT FOUND, JSON-объект с сообщением об ошибке \\\hline
\end{tabu}

\clearpage
Подробное описание тестов:

\begin{itemize}
  \item Сохранение файла в корневой директории сервера
  \begin{description}
    \item[Ресурс:] \url{/static}
    \item[HTTP-метод:] POST
    \item[Параметры запроса:] file: a.out, headers: "Content-type"="multipart/form-data"
    \item[Код, содержимое ответа:] 200, пустой JSON-объект "\{\}"
  \end{description}
  
  \item Сохранение файла в произвольной директории сервера
  \begin{description}
    \item[Ресурс:] \url{/static}
    \item[HTTP-метод:] POST
    \item[Параметры запроса:] query: "path=1\textbackslash 2\textbackslash 3\textbackslash4", file: a.out, headers: "Content-type"="multipart/form-data"
    \item[Код, содержимое ответа:] 200, пустой JSON-объект "\{\}"
  \end{description}
  
  \item Доступ к существующему файлу в корневой директории сервера
  \begin{description}
    \item[Ресурс:] \url{/static/a.out}
    \item[HTTP-метод:] GET
    \item[Код, содержимое ответа:] 200, содержимое файла a.out
    \item[Замечание:] содержимое ответа перенаправлено в файл a\_.out
  \end{description}
  
  \item Доступ к существующему файлу в произвольной директории сервера
  \begin{description}
    \item[Ресурс:] \url{/static/1\ 2\ 3\ 4\ a.out}
    \item[HTTP-метод:] GET
    \item[Код, содержимое ответа:] 200, содержимое файла a.out
  \end{description}
  
  \item Удаление существующего файла
  \begin{description}
    \item[Ресурс:] \url{/static/a.out}
    \item[HTTP-метод:] DELETE
    \item[Код, содержимое ответа:] 200, пустой JSON-объект "\{\}"
  \end{description}
  
  \item Удаление существующего файла в произвольной директории сервера
  \begin{description}
    \item[Ресурс:] \url{/static/1\ 2\ 3\ 4\ a.out}
    \item[HTTP-метод:] DELETE
    \item[Код, содержимое ответа:] 200, пустой JSON-объект "\{\}"
  \end{description}
  
  \item Удаление несуществующего файла
  \begin{description}
    \item[Ресурс:] \url{/static/a.out}
    \item[HTTP-метод:] DELETE
    \item[Код, содержимое ответа:] 404, JSON-объект "\{"error":"Not Found"\}"
    \item[Замечание:] файл "a.out" удалён с сервера в ходе предыдущих тестов
  \end{description}
  
  \item Удаление несуществующего файла в произвольной директории сервера
  \begin{description}
    \item[Ресурс:] \url{/static/1\ 2\ 3\ 4\ a.out}
    \item[HTTP-метод:] DELETE
    \item[Код, содержимое ответа:] 404, JSON-объект "\{"error":"Not Found"\}"
  \end{description}
  
  \item Доступ к несуществующему файлу в корневой директории сервера
  \begin{description}
    \item[Ресурс:] \url{/static/a.out}
    \item[HTTP-метод:] GET
    \item[Код, содержимое ответа:] 404, JSON-объект "\{"error":"Not Found"\}"
  \end{description}
  
  \item Доступ к несуществующему файлу в произвольной директории сервера
  \begin{description}
    \item[Ресурс:] \url{/static/1\ 2\ 3\ 4\ a.out}
    \item[HTTP-метод:] GET
    \item[Код, содержимое ответа:] 404, JSON-объект "\{"error":"Not Found"\}"
  \end{description}
\end{itemize}

\clearpage
\subsubsection{Тестирование надёжности и доступности серверов}
\noindent
\begin{tabu}{|L{1cm}|L{3cm}|L{2.5cm}|L{2cm}|X[l]|}\hline
  № & Описание теста & Входные данные & Ожидаемый результат & Полученный результат \\ \hline
  1 & Провести тест, в котором несколько пользователей одновременно загружают файл на сервер & Пользовательские файлы, их имена & Все файлы загружены по верным адресам & Каждый из пользователей успешно загрузил файл на сервер \\\hline
  2 & Провести тест, в котором происходит разрыв соединения в ходе загрузки файла & Пользовательский файл & Загрузка возобновляется по восстановлении соединения & Возникла ошибка. Необходима повторная загрузка файла. \\\hline
\end{tabu}

Отчёт об обнаруженных ошибках:

\noindent
\begin{tabu}{|l|X[l]|}\hline
  \multicolumn{2}{|l|}{Сервис: файловый сервер}\\\hline
  \multicolumn{2}{|l|}{Степень важности: средняя}\\\hline
  Надёжность & Сервис некорректно отрабатывает сценарий разрыва соединения \\\hline
\end{tabu}

\subsubsection{Выводы по результатам системного тестирования}
Сервис может быть использован только в средах, где сеть можно считать достаточно надёжной. Для использования в рамках сетей где часты разрывы необходимо производить доработку сервиса.

\subsubsection{Код теста}
\lstinputlisting[language=bash]{src/queries_st.sh}

\subsubsection{Выход терминала в ходе тестов}
\lstinputlisting{src/queries_st.log}


\subsection{Интеграционное тестирование}
Производится интеграционное тестирование подсистем работы с файлами, данными и подсистемы мониторинга.


Тест покрывает прецедент регистрации подсистем работы с файлами и данными в подсистеме мониторинга.
Тест не требует входных данных.


\subsubsection{Код теста}
\lstinputlisting[language=bash]{src/beacon_data_file_it.sh}

\subsubsection{Выход терминала в ходе тестов}
\lstinputlisting{src/beacon_data_file_it.log}

\end{document}